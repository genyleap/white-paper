\documentclass[a4paper,12pt,openany]{book}

% Essential packages for Russian document
\usepackage{polyglossia}
\setmainlanguage{russian}
\setotherlanguage{english}
\usepackage{fontspec}
\setmainfont{DejaVu Serif}
\newfontfamily\ipafont{Charis}
\usepackage{geometry}
\geometry{margin=2.0cm}
\usepackage{enumitem}
\usepackage{setspace}

% Spacing settings
\onehalfspacing
\setlength{\parskip}{0.2em}

% Remove blank pages before chapters
\let\cleardoublepage\clearpage

\begin{document}

% Титульная страница
\begin{titlepage}
    \begin{center}
        \vspace*{1.5cm}
        {\Huge \textbf{Белая книга Genyleap}} \\
        \vspace{0.5cm}
        {\Large Версия 1.0} \\
        \vspace{0.5cm}
        {\large Дата публикации: июнь 2025 года} \\
        \vspace{1.5cm}
        {\large Высококачественное ПО для лучшей жизни} \\
    \end{center}
    \vfill
\end{titlepage}

\chapter{Введение}
В Genyleap мы представляем будущее, в котором высококачественные, мощные и эффективные цифровые технологии служат всем. Мы разрабатываем простое и надежное ПО, улучшающее жизнь, бизнес и инновации. Благодаря открытым технологиям и блокчейну мы делаем передовые инструменты доступными, повышаем прозрачность и уважаем экологию с помощью устойчивого дизайна. Genyleap предлагает технологии, которые не только эффективны, но и вдохновляют и доступны каждому.

\chapter{Двигатель Cell: ядро технологий Genyleap}

Двигатель Cell (\texttt{Cell}) — это передовая система ПО с открытым кодом, которая составляет основу Genyleap. Создан по стандарту \texttt{C++23} ISO/IEC 14882:2024, он предоставляет современную платформу для разработки высококачественных, безопасных и производительных приложений.

Cell разработан для программистов, стремящихся к полному контролю, позволяя создавать что угодно — от настольных и мобильных приложений до веб-сервисов, сайтов и устройств \texttt{IoT} — без сложных или дорогих инструментов. Обеспечивает экономичную разработку без потери производительности.

\textbf{Основные характеристики:}
\begin{itemize}
    \item \textbf{Кроссплатформенность:} Cell работает нативно на ПК, мобильных, встроенных системах и \texttt{WebAssembly}, обеспечивая переносимость.
    \item \textbf{Модульность:} Компонентная структура позволяет настраивать и расширять функционал, оставаясь легковесной.
    \item \textbf{Энергоэффективность:} Cell минимизирует энергопотребление, поддерживая экологичное будущее.
    \item \textbf{Производительность и безопасность:} Оптимизирован для скорости и надежности, идеален для современных приложений.
    \item \textbf{Глобализация:} Поддержка многоязычности и интернационализации для глобального использования.
\end{itemize}

\footnote{Двигатель Cell поддерживает проекты (многоцелевой), работает на платформах от ПК до \texttt{IoT} (кроссплатформенный) и минимизирует энергопотребление (зеленые вычисления), сохраняя производительность.}

\chapter{Видение и стратегии}

Genyleap переопределяет цифровые технологии, чтобы расширить возможности разработчиков, ускорить инновации и способствовать устойчивости.

\begin{itemize}
    \item \textbf{Качественное ПО для всех:} Инструменты Genyleap доступны и эффективны для пользователей и разработчиков.
    \item \textbf{Поддержка бизнеса:} Genyleap помогает стартапам и компаниям превращать идеи в масштабируемые решения.
    \item \textbf{Экологичность:} Genyleap разрабатывает энергоэффективное ПО, сочетая прогресс с экологией.
\end{itemize}

\footnote{Блокчейн обеспечивает проверяемые, защищенные записи, укрепляя доверие и целостность системы.}

\chapter{Криптовалюта GENY (Токен)}
\begin{quote}
Мы создали не криптовалюту... мы создали ген! Это больше, чем токен; это ДНК взаимодействия в \texttt{Web3}.
\end{quote}
Токен GENY, официально \texttt{Genyleap}, — это цифровая валюта Genyleap, превращающая экосистему в децентрализованную сеть. GENY — это «цифровой ген», позволяющий создавать идентичность и влияние в \texttt{Web3}.
\begin{quote}
В каждом бите — ген... в каждом гене — мир!
\end{quote}

\section*{Официальное произношение}
На английском \textbf{GENY} произносится \textit{``Jenny''} ({\ipafont /ˈdʒe.ni/}). Название происходит от слов «ген» и «почему», задавая вопрос о творчестве в цифровом мире.

\section*{Технические характеристики GENY}
\begin{itemize}
    \item \textbf{Название токена}: \texttt{Genyleap}
    \item \textbf{Символ}: \texttt{GENY}
    \item \textbf{Стандарт}: \texttt{ERC-20}
    \item \textbf{Общий объем}: 256,000,000 единиц
    \item \textbf{Сжигаемый}: Да (с ограничениями \texttt{DAO})
    \item \textbf{Выкуп}: Да
    \item \textbf{Управление}: Через \texttt{Governor} в \texttt{DAO}
\end{itemize}
\vspace{-0.5em}
\begin{quote}
Раскрой свой ген! Будущее за теми, кто формирует свой цифровой ген через взаимодействие.
\end{quote}
\newpage

\section*{Особенности и аудитория GENY}
\subsection*{Особенности}
GENY расширяет возможности пользователей:
\begin{itemize}
    \item \textbf{Доступ к сервисам}: GENY — ключ к продуктам Genyleap, например, приложениям на Cell.
    \item \textbf{Награды за творчество}: GENY за качественный вклад (отзывы, контент, ПО).
    \item \textbf{Цифровая собственность}: GENY обеспечивает владение активами (\texttt{NFT}).
    \item \textbf{Быстрые транзакции}: Мгновенные платежи в экосистеме.
    \item \textbf{Управление}: Голосование в \texttt{DAO}.
    \item \textbf{Стимулы}: Награды за качество, а не спам.
\end{itemize}
\begin{quote}
Участвуй, и твой цифровой ген обретет форму!
\end{quote}
Например, пользователь может заработать 0.25 GENY за отзыв, а художник — 10 GENY за продажу \texttt{NFT}.

\subsection*{Аудитория GENY}
GENY для всех позитивно мыслящих:
\begin{itemize}
    \item \textbf{Пользователи}: Доступ к приложениям и награды за отзывы.
    \item \textbf{Разработчики}: Создание ПО с Cell.
    \item \textbf{Художники}: Создание и продажа цифровых активов.
    \item \textbf{Стартапы}: Превращение идей в продукты с Genyleap.
\end{itemize}
Каждый вклад в Genyleap ценен, и GENY связывает вас с экосистемой.
\newpage

\section*{Токеномика GENY}
GENY с объемом 256 млн единиц разработан для ликвидности, роста и устойчивости. Распределение стимулирует участие.

\begin{table}[h]
\centering
\caption{Распределение GENY (Часть 1)}
\small
\begin{tabular}{r c c c c c}
\hline
\textbf{Vesting} & \textbf{TGE (Млн)} & \textbf{\%} & \textbf{Токены (Млн)} & \textbf{Категория} & \textbf{№} \\
\hline
48 мес., 6 мес. клифф & 0 & 12.5 & 32 & Команда & 1 \\
36 мес., 6 мес. клифф & 0 & 6.3 & 16 & Инвесторы & 2 \\
24 мес. для 62 млн & 2 & 25.0 & 64 & Экосистема & 3 \\
По плану & 2 & 12.5 & 32 & Эйрдроп & 4 \\
\hline
\end{tabular}
\end{table}

\begin{table}[h]
\centering
\caption{Распределение GENY (Часть 2)}
\small
\begin{tabular}{r c c c c c}
\hline
\textbf{Vesting} & \textbf{TGE (Млн)} & \textbf{\%} & \textbf{Токены (Млн)} & \textbf{Категория} & \textbf{№} \\
\hline
Без vesting & 32 & 12.5 & 32 & Ликвидность & 5 \\
24 мес. для 28 млн & 4 & 12.5 & 32 & Казна и DAO & 6 \\
24 мес. для 28 млн & 4 & 12.5 & 32 & GenyLab & 7 \\
24 мес., 3 мес. клифф & 0 & 6.3 & 16 & Фонд роста & 8 \\
\hline
-- & 44 & 100.0 & 256 & Итого & -- \\
\hline
\end{tabular}
\end{table}

\textbf{Описание категорий:}
\begin{enumerate}
    \item \textbf{Команда}: Для управления и разработки.
    \item \textbf{Инвесторы}: Для финансовой поддержки.
    \item \textbf{Экосистема}: Награды пользователям и разработчикам.
    \item \textbf{Эйрдроп}: Привлечение новых пользователей.
    \item \textbf{Ликвидность}: Для бирж и торговли.
    \item \textbf{Казна и DAO}: Для управления и развития.
    \item \textbf{GenyLab}: Для инноваций и исследований.
    \item \textbf{Фонд роста}: Для новых проектов.
\end{enumerate}

\subsection*{Экономические детали}
\begin{itemize}
    \item \textbf{Общий объем}: 256 млн токенов GENY.
    \item \textbf{Первоначальный выпуск (\texttt{TGE})}: 44 млн (17.2\%).
    \item \textbf{Выкуп}: 15\% прибыли на выкуп в \texttt{buybackPool}.
    \item \textbf{Сжигание}: До 2\% в кризисах (взломы, низкая цена).
\end{itemize}
\newpage

\section*{Единицы и субъединицы GENY}
В Genyleap мы верим, что ценность кроется в малых единицах, а не в миллиардах токенов. Экосистема построена на стандартах \texttt{SI} и бинарной системе (степени 2), обеспечивая точность и прозрачность. Эта структура обеспечивает простой и уникальный опыт, где каждое взаимодействие, как ген, несет ценность.

\begin{quote}
Каждое взаимодействие — уникальный ген! GENY приносит чувство созидания.
\end{quote}
GENY — основная единица, с бинарными префиксами kibi (\texttt{Ki}) и mebi (\texttt{Mi}) для точности на всех уровнях.

\subsection*{Таблица единиц GENY}
\begin{table}[h]
\centering
\caption{Субъединицы GENY}
\small
\begin{tabular}{l c c r}
\hline
\textbf{Префикс} & \textbf{Символ} & \textbf{В GENY} & \textbf{В \texttt{субъединице}} \\
\hline
Mebi (\texttt{MiGENY}) & \texttt{MiGENY} & 1,048,576 & 268,435,456 \\
Kibi (\texttt{KiGENY}) & \texttt{KiGENY} & 1,024 & 262,144 \\
-- & \texttt{GENY} & 1 & 256 \\
Milli (\texttt{mGENY}) & \texttt{mGENY} & 0.001 & 0.000256 \\
Micro (\texttt{\textmu GENY}) & \texttt{\textmu GENY} & 0.000001 & 0.000000256 \\
Nano (\texttt{nGENY}) & \texttt{nGENY} & 0.000000001 & 0.000000000256 \\
Pico (\texttt{pGENY}) & \texttt{pGENY} & 0.000000000001 & 0.000000000000256 \\
\hline
\end{tabular}
\end{table}

\subsection*{Масштабирование по степеням 2}
Мы используем бинарные преобразования (например, 1,024 вместо 1,000) для точности. Например, 1 \texttt{KiGENY} = 1,024 GENY. Этот стандарт предотвращает ошибки округления в микротранзакциях (например, \texttt{IoT} или аналитика).

\subsection*{Применение GENY и GEN}
Мы используем милли, микро и нано для мелких взаимодействий и kibi, mebi для крупных проектов. Это упрощает экосистему Genyleap, особенно в чаевых.
\begin{itemize}
    \item \textbf{Микротранзакции}: Датчики \texttt{IoT} передают данные с 0.000000001 GENY.
    \item \textbf{Чаевые}: 0.1 GENY для поддержки поста.
    \item \textbf{Подписки}: Премиум-подписка за 10 GENY.
    \item \textbf{Взносы}: 2,000,000 GENY через \texttt{DAO}.
\end{itemize}

\textbf{Примеры:}
\begin{table}[h]
\centering
\caption{Примеры GENY и \texttt{GEN}}
\small
\begin{tabular}{l c r}
\hline
\textbf{Сценарий} & \textbf{В GENY} & \textbf{В \texttt{GEN}} \\
\hline
Грант DAO & 2,000,000 & 512,000,000 \\
Ликвидность & 500,000 & 128,000,000 \\
Премиум-подписка & 10 & 2,560 \\
Разблокировка & 1 & 256 \\
Большой чаевой & 0.5 & 128 \\
Малый чаевой & 0.1 & 25.6 \\
Микросервис & 0.01 & 2.56 \\
Аналитика & 0.0001 & 0.0256 \\
\texttt{IoT} сигнал & 0.000000001 & 0.000000256 \\
\hline
\end{tabular}
\end{table}

\subsection*{Привлекательные применения}
\begin{itemize}
    \item \textbf{\texttt{IoT} микротранзакции}: Датчики передают данные с 0.000000001 GENY.
    \item \textbf{Чаевые}: 0.1 GENY для поддержки контента.
    \item \textbf{Управление}: Голосование в \texttt{DAO} с 1,000 GENY.
    \item \textbf{Микроэкономика}: Услуги за 0.01 GENY.
\end{itemize}

\chapter{Аудитория Genyleap}
Genyleap создает ценность для:
\begin{itemize}
    \item \textbf{Пользователи}: Доступ к инструментам для задач.
    \item \textbf{Разработчики}: Платформа для создания ПО.
    \item \textbf{Художники}: Создание и продажа цифровых активов.
    \item \textbf{Стартапы}: Поддержка идей.
    \item \textbf{Организации}: Решения для бизнеса.
\end{itemize}

\chapter{Видение Genyleap}
Genyleap создает будущее, где ПО высококачественное, простое и устойчивое. С помощью открытого кода, блокчейна и \texttt{DAO} мы даем доступ к технологиям, улучшающим жизнь.

\chapter{Призыв к сотрудничеству}
Genyleap приглашает разработчиков, художников, стартапы и инвесторов к сотрудничеству. Подробности на genyleap.com или в соцсетях.

\section*{Адреса смарт-контрактов GENY}
Смарт-контракты GENY на Ethereum используют \LRE{UUPS} для управления транзакциями и децентрализованным управлением. Контракт \texttt{Multi-Sig} с тремя подписями обеспечивает безопасность и прозрачность. Адреса на \texttt{basescan.org}:

\begin{table}[h]
\centering
\caption*{Основные контракты GENY}
\small
\begin{tabular}{c r}
\hline
\textbf{Контракт} & \textbf{Адрес} \\
\hline
\texttt{GenyToken} & {\texttt{0x75d7a0e842a73c07847ee433c93d443dfea61038}} \\
\texttt{GenyAllocation (Proxy)} & {\texttt{0xFeEfB5200Bfd8A836964134b9B0Fe68d0e4F9dea}} \\
\texttt{Multi-Sin} & {\texttt{0x41EBc91077f37886CAc6aDEa67125A47c4d72930}} \\
\texttt{Creator} & {\texttt{0x7960F1B90b257BfF29D5164D16bca4C8030b7f6D}} \\
\hline
\end{tabular}
\end{table}

\section*{Ссылки}

\begin{enumerate}
    \item \textbf{Basescan}: \texttt{https://basescan.org/token/0x75d7a0e842a73c07847ee433c93d443dfea61038}
    \item \textbf{Исходный код}: \texttt{https://github.com/genyleap/geny-token}
\end{enumerate}

\end{document}