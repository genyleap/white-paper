\documentclass[a4paper,12pt,openany]{book}

% Essential packages for French document
\usepackage{polyglossia}
\setmainlanguage{french}
\setotherlanguage{english}
\usepackage{fontspec}
\setmainfont{DejaVu Serif}
\newfontfamily\ipafont{Charis}
\usepackage{geometry}
\geometry{margin=2.0cm}
\usepackage{enumitem}
\usepackage{setspace}

% Spacing settings
\onehalfspacing
\setlength{\parskip}{0.2em}

% Remove blank pages before chapters
\let\cleardoublepage\clearpage

\begin{document}

% Page de couverture
\begin{titlepage}
    \begin{center}
        \vspace*{1.5cm}
        {\Huge \textbf{Livre blanc de Genyleap}} \\
        \vspace{0.5cm}
        {\Large Version 1.0} \\
        \vspace{0.5cm}
        {\large Date de publication : Juin 2025} \\
        \vspace{1.5cm}
        {\large Des logiciels de haute qualité pour une vie meilleure} \\
    \end{center}
    \vfill
\end{titlepage}

\chapter{Introduction}
Chez Genyleap, nous imaginons un avenir où des technologies numériques de haute qualité, puissantes et efficaces servent à tous. Nous concevons des logiciels simples et fiables qui améliorent la vie quotidienne, les entreprises et l'innovation. Grâce aux technologies open-source et à la blockchain, nous rendons les outils avancés plus accessibles, renforçons la transparence et respectons l'environnement avec une conception durable. Genyleap propose une technologie non seulement efficace, mais aussi inspirante et accessible à tous.

\chapter{Moteur Cell : le cœur de la technologie Genyleap}

Le moteur Cell (\texttt{Cell}) est un système logiciel open-source avancé qui constitue le cœur de la fondation technologique de Genyleap. Développé selon la norme \texttt{C++23} ISO/IEC 14882:2024, il offre un cadre moderne pour la création d'applications de haute qualité, sécurisées et performantes.

Conçu pour les développeurs cherchant un contrôle total, Cell permet de construire n'importe quoi — des applications pour ordinateurs et mobiles aux services web, sites internet et appareils intelligents \texttt{IoT} — sans dépendre de chaînes d'outils complexes ou coûteuses. Il garantit un développement économique sans compromettre la performance ou la maintenabilité.

\textbf{Caractéristiques principales :}
\begin{itemize}
    \item \textbf{Compatibilité multiplateforme :} Cell fonctionne nativement sur PC, mobile, systèmes embarqués et \texttt{WebAssembly}, offrant une véritable portabilité entre appareils et systèmes d'exploitation.
    \item \textbf{Architecture modulaire et légère :} Sa structure basée sur des composants permet une personnalisation et une extensibilité maximales tout en restant légère et efficace.
    \item \textbf{Efficacité énergétique :} Conçu dans une optique de durabilité, Cell minimise la consommation d'énergie, soutenant un avenir numérique plus écologique.
    \item \textbf{Haute performance et sécurité :} Optimisé pour la vitesse et la sécurité, Cell offre un comportement système fiable, idéal pour les applications modernes nécessitant réactivité et résilience.
    \item \textbf{Support de la mondialisation :} Avec des fonctionnalités multilingues et un support d'internationalisation, Cell est prêt pour un déploiement mondial.
\end{itemize}

\footnote{Le moteur Cell soutient divers projets (multifonction), fonctionne efficacement sur des plateformes allant des PC aux \texttt{IoT} (multiplateforme), et est conçu pour une faible consommation d'énergie (informatique verte), réduisant son empreinte environnementale tout en offrant des performances robustes.}

\chapter{Vision et stratégies}

Genyleap redéfinit la technologie numérique pour autonomiser les développeurs, accélérer l'innovation et promouvoir la durabilité dans l'industrie logicielle.

\begin{itemize}
    \item \textbf{Logiciels de haute qualité pour tous :} Les outils de Genyleap sont accessibles, intuitifs et efficaces, servant à la fois les utilisateurs quotidiens et les développeurs professionnels avec des capacités performantes.
    \item \textbf{Autonomisation des entreprises :} Grâce à des outils puissants et des conseils experts, Genyleap aide les startups et les entreprises à transformer leurs idées en solutions évolutives et réussies.
    \item \textbf{Durabilité environnementale :} Genyleap favorise l'informatique verte en concevant des logiciels à faible consommation et haute efficacité, alignant progrès technologique et responsabilité écologique.
\end{itemize}

\footnote{La blockchain garantit des enregistrements de transactions vérifiables et inviolables, renforçant la confiance numérique et l'intégrité du système.}

\chapter{Cryptomonnaie GENY (Jeton)}
\begin{quote}
Nous n'avons pas créé une cryptomonnaie… nous avons créé un gène ! Ce n'est pas qu'un jeton ; c'est l'ADN de l'interaction et de la création dans \texttt{Web3}.
\end{quote}
Le jeton GENY, officiellement nommé \texttt{Genyleap}, est la monnaie numérique de Genyleap, transformant notre écosystème en un réseau dynamique et décentralisé. GENY est un « gène numérique », permettant aux individus de créer leur identité et leur impact dans \texttt{Web3}.
\begin{quote}
Dans chaque bit, un gène… dans chaque gène, un monde !
\end{quote}

\section*{Prononciation officielle}
En anglais, \textbf{GENY} se prononce \textit{``Jenny''} ({\ipafont /ˈdʒe.ni/}). Le nom dérive de la fusion entre « gène » et « pourquoi », posant une question conceptuelle sur l'origine de la créativité dans la structure génétique du monde numérique.

\section*{Spécifications techniques de GENY}
\begin{itemize}
    \item \textbf{Nom du jeton/cryptomonnaie}: \texttt{Genyleap}
    \item \textbf{Symbole}: \texttt{GENY}
    \item \textbf{Standard}: \texttt{ERC-20}
    \item \textbf{Offre totale}: 256,000,000 unités
    \item \textbf{Brûlable}: Oui (avec des restrictions définies par l'équipe et \texttt{DAO})
    \item \textbf{Rachat}: Oui
    \item \textbf{Gouvernance}: Via des contrats intelligents \texttt{Governor} dans \texttt{DAO}
\end{itemize}
\vspace{-0.5em}
\begin{quote}
Libérez votre gène ! L'avenir appartient à ceux qui façonnent leur gène numérique par la pensée et l'interaction.
\end{quote}
\newpage

\section*{Caractéristiques clés et public cible de GENY}
\subsection*{Caractéristiques clés}
GENY est conçu pour autonomiser les utilisateurs, offrant les fonctionnalités suivantes :
\begin{itemize}
    \item \textbf{Accès aux services}: GENY est la clé pour utiliser les produits Genyleap, comme les applications basées sur le moteur Cell.
    \item \textbf{Récompenses créatives}: Les utilisateurs gagnent des GENY pour des contributions de qualité (ex. retours, contenu, logiciels).
    \item \textbf{Propriété numérique}: GENY garantit la propriété sécurisée des actifs numériques (ex. \texttt{NFT}).
    \item \textbf{Transactions rapides}: Paiements instantanés à faible coût dans l'écosystème.
    \item \textbf{Gouvernance décentralisée}: Vote dans le \texttt{DAO} pour les décisions du projet.
    \item \textbf{Structure d'incitation}: Récompenses basées sur la qualité, pas sur les activités spam.
\end{itemize}
\begin{quote}
Participez, et votre gène numérique prend forme !
\end{quote}
Par exemple, un utilisateur peut gagner 0.25 GENY pour un retour d'expérience, ou un artiste peut vendre un \texttt{NFT} et recevoir 10 GENY en récompense.

\subsection*{Public cible de GENY}
GENY s'adresse à tous les esprits positifs, pas seulement aux développeurs ou investisseurs :
\begin{itemize}
    \item \textbf{Utilisateurs généraux}: Accès aux applications Genyleap et récompenses pour les retours.
    \item \textbf{Développeurs}: Création de logiciels innovants avec le moteur Cell.
    \item \textbf{Artistes}: Création et vente d'actifs numériques.
    \item \textbf{Startups}: Transformation des idées en produits avec le soutien de Genyleap.
\end{itemize}
Chaque idée ou contribution dans Genyleap a de la valeur, et GENY vous connecte à l'écosystème.
\newpage

\section*{Tokenomics de GENY}
Avec une offre totale de 256 millions d'unités, GENY est conçu pour assurer liquidité, croissance et durabilité. La distribution des jetons est planifiée pour encourager la participation.

\begin{table}[h]
\centering
\caption{Répartition des jetons GENY (Partie 1)}
\small
\begin{tabular}{r c c c c c}
\hline
\textbf{Vesting} & \textbf{TGE (Mln)} & \textbf{\%} & \textbf{Jetons (Mln)} & \textbf{Catégorie} & \textbf{N°} \\
\hline
48 mois, 6 mois de blocage & 0 & 12.5 & 32 & Équipe & 1 \\
36 mois, 6 mois de blocage & 0 & 6.3 & 16 & Investisseurs & 2 \\
24 mois pour 62 mln & 2 & 25.0 & 64 & Écosystème & 3 \\
Par plan & 2 & 12.5 & 32 & Airdrop & 4 \\
\hline
\end{tabular}
\end{table}

\begin{table}[h]
\centering
\caption{Répartition des jetons GENY (Partie 2)}
\small
\begin{tabular}{r c c c c c}
\hline
\textbf{Vesting} & \textbf{TGE (Mln)} & \textbf{\%} & \textbf{Jetons (Mln)} & \textbf{Catégorie} & \textbf{N°} \\
\hline
Sans vesting & 32 & 12.5 & 32 & Liquidité & 5 \\
24 mois pour 28 mln & 4 & 12.5 & 32 & Trésorerie et DAO & 6 \\
24 mois pour 28 mln & 4 & 12.5 & 32 & GenyLab & 7 \\
24 mois, 3 mois de blocage & 0 & 6.3 & 16 & Fonds de croissance & 8 \\
\hline
-- & 44 & 100.0 & 256 & Total & -- \\
\hline
\end{tabular}
\end{table}

\textbf{Descriptions des catégories :}
\begin{enumerate}
    \item \textbf{Équipe}: Pour la gestion et le développement du projet.
    \item \textbf{Investisseurs}: Pour le soutien financier initial.
    \item \textbf{Écosystème}: Récompenses pour les contributeurs aux services.
    \item \textbf{Airdrop}: Distribution de jetons pour attirer de nouveaux utilisateurs.
    \item \textbf{Liquidité}: Fournir de la liquidité sur les échanges.
    \item \textbf{Trésorerie et DAO}: Soutien à la prise de décision décentralisée et au développement futur.
    \item \textbf{GenyLab}: Financement de projets innovants et de recherches.
    \item \textbf{Fonds de croissance}: Financement de nouveaux projets et expansion.
\end{enumerate}

\subsection*{Détails économiques}
\begin{itemize}
    \item \textbf{Offre totale}: 256 millions de jetons GENY.
    \item \textbf{Émission initiale (\texttt{TGE})}: 44 millions de jetons (17.2\%).
    \item \textbf{Rachat de jetons}: 15\% des profits pour le rachat et transfert au \texttt{buybackPool}.
    \item \textbf{Brûlage de jetons}: Jusqu'à 2\% de l'offre circulante brûlée annuellement en cas de crise (ex. piratage ou prix très bas).
\end{itemize}
\newpage

\section*{Unités et sous-unités GENY}
Chez Genyleap, nous croyons que la vraie valeur réside dans les plus petites unités, pas dans des milliards de jetons inefficaces. Notre écosystème repose sur les normes \texttt{SI} et un système binaire (puissances de 2), garantissant précision et transparence à tous les niveaux. Cette structure intelligente offre une expérience simple et unique, notamment dans les systèmes de récompenses et microtransactions.

\begin{quote}
Chaque interaction, un gène unique ! GENY apporte un sentiment de création et de valeur inégalé.
\end{quote}
GENY est l'unité principale de valeur, utilisant des préfixes binaires comme kibi (\texttt{Ki}) et mebi (\texttt{Mi}) pour maintenir la précision.

\subsection*{Tableau des unités GENY}
\begin{table}[h]
\centering
\caption{Sous-unités GENY}
\small
\begin{tabular}{l c c r}
\hline
\textbf{Préfixe} & \textbf{Symbole} & \textbf{Montant en GENY} & \textbf{Montant en \texttt{sous-unité}} \\
\hline
Mebi (\texttt{MiGENY}) & \texttt{MiGENY} & 1,048,576 & 268,435,456 \\
Kibi (\texttt{KiGENY}) & \texttt{KiGENY} & 1,024 & 262,144 \\
-- & \texttt{GENY} & 1 & 256 \\
Milli (\texttt{mGENY}) & \texttt{mGENY} & 0.001 & 0.000256 \\
Micro (\texttt{\textmu GENY}) & \texttt{\textmu GENY} & 0.000001 & 0.000000256 \\
Nano (\texttt{nGENY}) & \texttt{nGENY} & 0.000000001 & 0.000000000256 \\
Pico (\texttt{pGENY}) & \texttt{pGENY} & 0.000000000001 & 0.000000000000256 \\
\hline
\end{tabular}
\end{table}

\subsection*{Échelonnage binaire}
Nous utilisons des conversions binaires (puissances de 2, ex. 1,024 au lieu de 1,000) pour garantir la précision. Par exemple, 1 \texttt{KiGENY} = 1,024 GENY. Ce standard évite les erreurs d'arrondi dans les microtransactions (ex. \texttt{IoT} ou analyses de données).

\subsection*{Applications de GENY et GEN}
Nous utilisons des unités milli, micro et nano pour les petites interactions, et kibi, mebi pour les grands projets. Cette approche simplifie l'écosystème Genyleap, notamment dans les systèmes de pourboires.
\begin{itemize}
    \item \textbf{Microtransactions}: Capteurs \texttt{IoT} avec 0.000000001 GENY.
    \item \textbf{Pourboires sociaux}: 0.1 GENY pour soutenir un post créatif.
    \item \textbf{Abonnements}: Abonnement premium avec 10 GENY.
    \item \textbf{Contributions majeures}: 2,000,000 GENY via \texttt{DAO} pour des projets communautaires.
\end{itemize}

\textbf{Exemples d'utilisation :}
\begin{table}[h]
\centering
\caption{Exemples d'utilisation de GENY et \texttt{GEN}}
\small
\begin{tabular}{l c r}
\hline
\textbf{Scénario} & \textbf{Équivalent en GENY} & \textbf{Montant en \texttt{GEN}} \\
\hline
Subvention DAO & 2,000,000 & 512,000,000 \\
Injection de liquidité & 500,000 & 128,000,000 \\
Abonnement premium & 10 & 2,560 \\
Déblocage fonction & 1 & 256 \\
Gros pourboire & 0.5 & 128 \\
Petit pourboire & 0.1 & 25.6 \\
Appel microservice & 0.01 & 2.56 \\
Analyse de données & 0.0001 & 0.0256 \\
Signal \texttt{IoT} & 0.000000001 & 0.000000256 \\
\hline
\end{tabular}
\end{table}

\subsection*{Applications attrayantes}
\begin{itemize}
    \item \textbf{Microtransactions \texttt{IoT}}: Capteurs envoyant des données avec 0.000000001 GENY.
    \item \textbf{Pourboires créatifs}: 0.1 GENY pour soutenir un contenu inspirant.
    \item \textbf{Gouvernance dynamique}: Vote dans \texttt{DAO} avec 1,000 GENY.
    \item \textbf{Micro-économie}: Services numériques pour 0.01 GENY.
\end{itemize}

\chapter{Public cible de Genyleap}
Genyleap crée de la valeur pour :
\begin{itemize}
    \item \textbf{Utilisateurs généraux}: Outils pratiques pour les tâches quotidiennes.
    \item \textbf{Développeurs}: Plateforme pour créer des logiciels innovants.
    \item \textbf{Artistes}: Création et vente d'actifs numériques.
    \item \textbf{Startups}: Soutien pour transformer des idées en produits.
    \item \textbf{Organisations}: Solutions sur mesure pour les entreprises.
\end{itemize}

\chapter{Vision de Genyleap}
Genyleap construit un avenir où les logiciels numériques sont de haute qualité, simples et durables. Avec des technologies open-source, la blockchain et \texttt{DAO}, nous offrons un accès universel à la technologie.

\chapter{Appel à la collaboration}
Genyleap invite les développeurs, artistes, startups et investisseurs à nous rejoindre dans ce voyage innovant. Pour plus d'informations, visitez genyleap.com ou contactez-nous sur les réseaux sociaux.

\section*{Adresses des contrats intelligents GENY}
Les contrats intelligents GENY sont déployés sur la blockchain Ethereum et utilisent l'architecture \LRE{UUPS} pour gérer les transactions, l'allocation des jetons et la gouvernance décentralisée de l'écosystème Genyleap. Le contrat \texttt{Multi-Sig} avec une structure à trois signatures garantit une sécurité et une transparence accrues. Consultez les adresses sur \texttt{basescan.org} pour les détails :

\begin{table}[h]
\centering
\caption*{Contrats principaux GENY}
\small
\begin{tabular}{c r}
\hline
\textbf{Nom du contrat} & \textbf{Adresse du contrat} \\
\hline
\texttt{GenyToken} & {\texttt{0x2a3d6f8c1fc4AcDcf3A75d19b445bae02F03676B}} \\
\texttt{GenyAllocation (Proxy)} & {\texttt{0xEeef3EFb77C72aC202BFfd7AFE123e9b86c15360}} \\
\texttt{Multi-Sig} & {\texttt{0x1a0819A7412BbFed6322C8B498aa58E3BD4d53B4}} \\
\texttt{Créateur} & {\texttt{0x477A5692e3D72a15eC3657A66F1F0bE67dAEA8B1}} \\
\hline
\end{tabular}
\end{table}

\section*{Références et adresses}

\begin{enumerate}
    \item \textbf{Basescan}: \texttt{https://basescan.org/token/0x2a3d6f8c1fc4AcDcf3A75d19b445bae02F03676B}
    \item \textbf{Adresse source}: \texttt{https://github.com/genyleap/geny-token}
\end{enumerate}

\end{document}
